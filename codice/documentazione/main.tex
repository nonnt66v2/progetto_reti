\documentclass[12pt,italian,a4paper,oneside,openright]{book}
\usepackage{url,amsfonts,epsfig}
\usepackage[italian]{babel}
\usepackage[utf8]{inputenc}
\usepackage{vmargin}
\usepackage{amsmath}
\usepackage{indentfirst}
\usepackage{graphicx}
\graphicspath{{img/}}
\usepackage[hyperindex]{hyperref} %per l'indice interattivo
\hypersetup{colorlinks=true, linkcolor=black} %per colorare i link
\DeclareGraphicsRule{.jpg}{jpg}{}{} %da commentare per il PDF
\setmarginsrb{35mm}{30mm}{30mm}{30mm}{0mm}{10mm}{0mm}{10mm}

%\documentclass{article}
\usepackage{listings}
\usepackage{xcolor}


% Impostazioni per lo stile del codice Java
\lstset{
	language=Java,
	basicstyle=\ttfamily\small,
	keywordstyle=\color{blue},
	commentstyle=\color{green!60!black},
	stringstyle=\color{orange},
	numbers=left,
	numberstyle=\tiny,
	stepnumber=1,
	numbersep=5pt,
	backgroundcolor=\color{gray!5},
	showspaces=false,
	showstringspaces=false,
	showtabs=false,
	frame=single,
	rulecolor=\color{black},
	tabsize=4,
	captionpos=b,
	breaklines=true,
	breakatwhitespace=false,
	title=\lstname,
	escapeinside={\%}{)},
	morekeywords={*,...}
}

\title{Template per la tesina in .tex}
\author{Lorenzo Di Palo}
\date{29/02/2024}

\begin{document}
	\pagenumbering{Roman}
	
	%%%% Opzione per interlinea 2
	\baselineskip 1.5em
	
	%% FRONTESPIZIO
	{ \thispagestyle{empty}
		
		
		\vskip 1cm \large \centerline{\textsc{Università degli Studi di
				Napoli ``Parthenope''}}
		
		\centerline {\textsc{Facoltà di Scienze e Tecnologie}}
		
		\centerline {\small\textsc{Corso di laurea in Informatica}}
		
	
		
		\vskip 0.5cm
		
		\large \centerline {\textsc{Documentazione Reti di Calcolatori}}
		
		\vskip 0.5cm
		
		\Large \centerline {Traccia:\\Università}
		
		
		\vskip 4.5cm
		
		
		\large
		\begin{minipage}[t]{7cm}
			\textsc{Docente}
			
			Emanuel Di Nardo\\
			
		\end{minipage}
		\hfill
		\begin{tabular}[t]{l}
			\textsc{Candidato} \\
			Di Palo Lorenzo\\
			Matricola: 0124002580
		\end{tabular}
		
		\vskip 2.0 cm \Large \centerline {Anno Accademico 2023-2024}
		\vfill \eject}
	
	\tableofcontents
	
	\newpage
	
	\pagenumbering{arabic}
	\thispagestyle{headings}
\chapter{Introduzione}
\label{key}
\section{Obiettivo}
Realizzare un'applicazione client/server parallelo per gestire gli esami \\universitari

\paragraph{Studente}
\begin{itemize}
	\item Chiede alla segreteria se ci siano esami disponibili per un corso
	\item Invia una richiesta di prenotazione di un esame alla segreteria
\end{itemize}


\paragraph{Segreteria}
\begin{itemize}
	\item Inserisce gli esami sul server dell'università (salvare in un file o conservare in
	memoria il dato)
	\item Inoltra la richiesta di prenotazione degli studenti al server universitario
	\item Fornisce allo studente le date degli esami disponibili per l'esame scelto dallo
	studente
\end{itemize}

\paragraph{Server Universitario}
\begin{itemize}
	\item Riceve l'aggiunta di nuovi esami
	\item Riceve la prenotazione di un esame
\end{itemize}




\thispagestyle{headings} % Applica lo stile di intestazione alla pagina

	\newpage
	\pagestyle{plain}
	
	
\end{document}
