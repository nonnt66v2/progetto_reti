\chapter{Descrizione dell'architettura} 
\newpage

\section{Architettura}
	Il sistema è costituito da 2 architetture client/server
	(I/O Multiplexing)
	\begin{itemize}
		\item \textbf{Server/Segreteria:} La prima connessione che avviene è questa.\\
		Quando viene eseguito lo script \texttt{start.sh}\\
		SERVER - SEGRETERIA - STUDENTE (\ref{studente}) vengono avviati\\
		il $Server$ attende che la $Segreteria$ si connetta.
		\item \textbf{Segreteria/Studente:} La seconda connessione avviene dopo che la prima non ha riportato errori.\\
		Se quest'ultima non riporta errori, \\
		$Segreteria$ attende l'inserimento delle Credenziali da parte dello $Studente$.
	\end{itemize}
	\begin{center}
		\includegraphics[scale=0.7]{img/schema\_applicazione.png}
	\end{center}
	\newpage
\subsection{Componenti dell'architettura}
\begin{itemize}
	\item \textbf{Studente:}\label{studente} Fornisce interfaccia a linea di comando per:\\
	\begin{itemize}
	\item \textbf {Visualizza Appelli:}\\ Permette allo $Studente$ di scegliere tra la visualizzazione di tutti gli appelli nel suo corso o di un appello specifico.
	\item \textbf {Prenota Appello:}\\
	Richiede all'utente il nome del codice dell'esame al quale vuole prenotarsi e lo prenota nel primo appello disponibile.
	\item \textbf {Logout:}\\
	elimina le informazioni della sessione
	\end{itemize}
	
\end{itemize}

\begin{itemize}
	\item \textbf{Segreteria:}\label{segreteria}  Fornisce interfaccia a linea di comando per:
	\begin{itemize}
		\item \textbf{Gestire Studenti:}\\
		Gestisce le Richieste da parte di $Studente$
		\item \textbf{Inserire Appelli:}
		Inserisce nuovi appelli per esami esistenti in $Server$
	\end{itemize}
\end{itemize}

\begin{itemize}
\item \textbf{Server:}\label{server} Fornisce a $Segreteria$ la possibilità di:
	\begin{itemize}
		\item \textbf{Aggiunta Appello:}\\
		\item \textbf{Aggiunta Prenotazione:}\\
	\end{itemize}
\end{itemize}

\subsection{Flusso di Comunicazione}
 Server: Attende connessioni.\\
 Segreteria: Dopo aver stabilito una connessione con $Server$, $Segreteria$ è pronta a ricevere nuove connessioni.\\
 Studente: Dopo aver stabilito una connessione, $Studente$ invia le credenziali a $Segreteria$ per l'autenticazione. \\
 Una volta autenticato, $Studente$ può eseguire le Operazioni descritte in \ref{studente}.\\
 
 
