\newpage
\section{Segreteria}
\subsection{Connessione al Server Universitario}
La connessione al server universitario rispecchia la connessione avvenuta in \ref{lst:ccs}

\subsection{Hosting}
La connessione da parte di $Studente$ richiede una socket che rimanga in ascolto in attesa di nuove connessioni\\
In questo caso la socket è "listenfd"\\
 \begin{lstlisting}[caption=Codice Ascolto Socket, label=lst:cass]
 	listenfd = getListenSocket(AF_INET, SOCK_STREAM, 0);
 	secaddr.sin_family = AF_INET;
 	secaddr.sin_addr.s_addr = htonl(INADDR_ANY);
 	secaddr.sin_port = htons(PORT_SERVER);
 	bindListener(listenfd, secaddr, sizeof(secaddr));
 	/**
 	* Mettiamo il server in ascolto, specificando quante connessioni possono essere in attesa venire accettate
 	* tramite il secondo argomento della chiamata.
 	*/
 	if ((listen(listenfd, 5)) < 0) {
 		perror("Errore nell'operazione di listen!");
 		exit(1);
 	}
 \end{lstlisting}
 \newpage
\subsection{Connessione al DB "UNIVERSITA"}
Se la connessione va a buon fine, $conn$ deterrà l'entry point per effettuare query sul DB\\
\begin{lstlisting}[caption=Codice Connessione DB, label=lst:ccdb]
	conn = mysql_init(NULL);
	if (conn == NULL) {
		fprintf(stderr, "mysql_init() fallita\n");
		exit(1);
	}
	if (mysql_real_connect(conn, "localhost", "nonnt66", "password", "universita", 3306, NULL, 0) == NULL) {
		fprintf(stderr, "mysql_real_connect() fallita: %s\n", mysql_error(conn));
		mysql_close(conn);
		exit(1);
	} else {
		printf("Connessione al database avvenuta con successo\n");
	}
\end{lstlisting}
\newpage


\subsection{Descrittori}
\begin{itemize}
\item Dichiarazione Descrittori:
\begin{itemize}
	Identifico 3 set di descrittori:
	\item \textbf{master\_set: } l'insieme di descrittori che verrà passato alla funzione select, settato inizialmente a zero.
	\item \textbf{read\_set: } l'insieme di descrittori che
	verifica quale descrittore è pronto in lettura
	\item \textbf{write\_set: } l'insieme di descrittori che
	verifica quale descrittore è pronto in scrittura\\
\end{itemize}
Inizialmente i vene impostato il massimo descrittore, scegliendo, tra la socket dello $Studente$ e la socket del $Server$
\begin{lstlisting}[caption=Codice Descrittori, label=lst:cd]
	fd_set read_set, write_set, master_set;
	int max_fd;
	FD_ZERO(&master_set);
	
	FD_SET(sockfd, &master_set);
	max_fd = sockfd;
	
	FD_SET(listenfd, &master_set);
	max_fd = max(max_fd, listenfd);
\end{lstlisting}
\newpage
\item Selezione Descrittori:
Si impostano i descrittori al valore della master\_set e si esegue la select per vedere se ci sono connessioni pronte
\begin{lstlisting}[caption=Codice Descrittori Select, label=lst:cds]
	if (select(max_fd + 1, &read_set, &write_set, NULL, NULL) < 0) {
		perror("Errore nell'operazione di select!");
	}
\end{lstlisting}
\item Verifica Descrittore Pronto:
Vado a verificare quale è il descrittore pronto in lettura/scrittura rispettivamente lato $Studente$, $Server$
 \begin{lstlisting}[caption=Codice Accettazione Studente - Lettura Scelta Studente, label=lst:caslss]
  if (FD_ISSET(listenfd, &read_set)) {
 	/**
 	* La system call accept permette di accettare una nuova connessione (lato server) in entrata da un client.
 	*/
 	if ((client_sockets[dim].connfd = accept(listenfd, (struct sockaddr *) NULL, NULL)) < 0) {
 		perror("Errore nell'operazione di accept!");
 	} else {
 		/**
 		* Si aggiunge il descrittore legato alla nuova connessione da uno studente all'interno dell'array di
 		* descrittori master_set e si ricalcola il numero di posizioni da controllare nella select.
 		*/
 		FD_SET(client_sockets[dim].connfd, &master_set);
 		max_fd = max(max_fd, client_sockets[dim].connfd);
 		/*** esegue oprazioni SQL ***/
 		/*** Itera sui client connessi ***/
 		if (FD_ISSET(client_sockets[i].connfd, &read_set) && client_sockets[i].connfd != -1) {
 			read(client_sockets[i].connfd, &behaviour, sizeof(behaviour));
 \end{lstlisting}
 
\end{itemize}




