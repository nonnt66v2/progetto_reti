\chapter{Dettagli implementativi dei client/server} 
\newpage


\section{Studente}
\subsection{Connessione alla Segreteria}
$Studente$ stabilisce una connessione con $Segreteria$ utilizzando una socket, specificandone\\
Dominio($AF\_INET$)\\
Tipo ($SOCK\_STREAM$)\\
Protocollo (0)\\
\begin{lstlisting}[caption=Codice Connessione Socket, label=lst:ccs]
	
	if ((sockfd = socket(AF_INET, SOCK_STREAM, 0)) < 0) {
		perror("Errore nella creazione della socket!");
		exit(1);
	}
	servaddr.sin_family = AF_INET;
	if (inet_pton(AF_INET, argv[1], &servaddr.sin_addr) <= 0) {
		fprintf(stderr, "Errore inet_pton per %s\n", argv[1]);
		exit(1);
	}
	servaddr.sin_port = htons(PORT_CLIENT);
	
	if (connect(sockfd, (struct sockaddr *) &servaddr, sizeof(servaddr)) < 0) {
		perror("Errore nella connect: ");
		printf("\n");
		exit(1);
	}
\end{lstlisting}
\newpage
\subsection{Autenticazione}
$Studente$ inserisce credenziali e se non sono presenti nel database viene terminata l'esecuzione.
\begin{lstlisting}	[caption=Codice Autenticazione Segreteria, label=lst:cas]
	 printf("LOGIN\n");
	printf("Inserire matricola: ");
	scanf("%d", &mat);
	
	/**
	* Pulisco il buffer di input.
	*/
	while ((c = getchar()) != '\n' && c != EOF);
	
	printf("Inserire password: ");
	fgets(pass, sizeof(pass), stdin);
	pass[strlen(pass) - 1] = 0;
	
	write(sockfd, &mat, sizeof(mat));
	write(sockfd, pass, sizeof(pass));
	
	char state[255] = {0};
	read(sockfd, state, sizeof(state));
	
	printf("\nEsito login: %s", state);
	printf("\n");
	
	if (strcmp(state, "credenziali non corrette, accesso negato!") == 0) {
		exit(1);
	}
	
\end{lstlisting}
\newpage
\subsection{Scelta Operazione}
$Studente$ può effettuare una delle scelte riportate in \ref{studente}
\begin{lstlisting}[caption=Codice Scelta Operazione, label=lst:cso]
	printf("\nInserire il numero relativo all'operazione che si vuole effettuare:\n");
	printf("1 - Visualizza appelli disponibili\n");
	printf("2 - Prenota un appello\n");
	printf("3 - Logout\n");
	printf("Scelta: ");
	scanf("%d", &request);
	printf("\n");
	/*** pulisce buffer ***/
	write(sockfd, &request, sizeof(request));
	
\end{lstlisting}
\newpage
\subsection{Ottieni Risposta}
In base alla scelta effettuata, si ottiene una risposta dalla $Segreteria$. (Tentativo Formattazione Testo in C vano)
\begin{lstlisting}[caption=Codice Ottieni Risposta (Richiesta Appelli), label=lst:cor]
	read(sockfd, &num_rows, sizeof(num_rows));
	
	if (num_rows == 0) {
		printf("\nNon esistono appelli disponibili!\n");
	} else {
		printf("\nAppelli disponibili:\n\n");
		
		printf("| ID\tNome esame\tData | \n\n ");
		for (int i = 0; i < num_rows; i++) {
			read(sockfd, &id, sizeof(id));
			read(sockfd, name, sizeof(name));
			read(sockfd, date, sizeof(date));
			printf("%d\t%s\t%s\n", id, name, date);
			printf("-------------------------------\n");
		}
	}	
\end{lstlisting}


